\chapter{Logic}

Logic provides the materials needed for a rigorous theory of how 
to prove theorems and build axiomatic systems such as axiomatic 
set theory. It contains fascinating theorems with widespread 
implications such as G\"odel's infamous Incompleteness Theorems. 
The information in this chapter is based on Shoenfield's classic 
\emph{Mathematical Logic} (1967).

%%%%%%%%%%%%%%%%%%%%%%%%%%%%%%%%%%%%%%%%%%%%%%%%%%%%%%%%%%%%%%%%%%

\newpage

\section{Notation}

\begin{itemize}
	\item \bt{u} and \bt{v} are syntactical variables which vary 
	through all expressions.
	
	\item \bt{A}, \bt{B}, \bt{C}, and \bt{D} are syntactical 
	variables which vary through formulas.
	
	\item \bt{x}, \bt{y}, \bt{z}, and \bt{w} are syntactical 
	variables which vary through variables.
	
	\item \bt{f} and \bt{g} are syntactical variables which vary 
	through function symbols.
	
	\item \bt{p} and \bt{q} are syntactical variables which vary 
	through predicate symbols.
	
	\item \bt{e} is a syntactical variables which varies through 
	constants.
	
	\item \bt{a}, \bt{b}, \bt{c}, and \bt{d} are syntactical  
	variables which vary through terms.
	
	\item $\bt{b}_{\nbts{x}}[\nbts{a}]$ 
	designates the expression obtained from \bt{b} by replacing 
	all occurrences of $\nbts{x}$ by $\nbts{a}$ respectively.
	
	\item $\bt{A}_{\nbts{x}}[\nbts{a}]$ 
	designates the expression obtained from \bt{A} by replacing 
	all free occurrences of $\nbts{x}$ by $\nbts{a}$ respectively.
	
	\item \bt{i} and \bt{j} are syntactical variables which vary 
	through names.
\end{itemize}

%%%%%%%%%%%%%%%%%%%%%%%%%%%%%%%%%%%%%%%%%%%%%%%%%%%%%%%%%%%%%%%%%%

\newpage

\section{Functions and Predicates}

Main Idea: Functions assign a collection of elements from one set 
to a single element of another set. Predicates represent some 
relationship between a collection of elements from a set.

\begin{definition}[$n$-tuple]
	An \df{$n$-tuple}\index{logic!$n$-tuple} in $A$ is a sequence 
	of $n$ (not necessarily	distinct) objects in $A$.
\end{definition}

We write $(a_1, a_2, \cdots, a_n)$ for the $n$-tuple consisting of 
the objects $a_1, a_2, \cdots, a_n$ in that order. We agree that 
there is exactly one 0-tuple in $A$, and we designate it by ( ).

\begin{definition}[Functions]
	A mapping from the set of $n$-tuples in $A$ to $B$ is called an 
	\df{$n$-ary function from $A$ to $B$}\index{logic!function}.
\end{definition}

\begin{example}[Truth Functions]
	A \df{truth function}\index{logic!truth function} is a function 
	from the set of truth values, $\{ \bt{T}, \bt{F} \}$, to the 
	set of truth values. We have
	\begin{itemize}
		\item the \df{and} truth function:
		\begin{align*}
			H_\land(\bt{T},\bt{T}) &= \bt{T}\,, \\
			H_\land(\bt{T},\bt{F}) &= H_\land(\bt{F},\bt{T})
				= H_\land(\bt{F},\bt{F}) = \bt{F}\,.
		\end{align*}
		\item the \df{or} truth function:
		\begin{align*}
			H_\lor(\bt{T},\bt{T}) &= H_\lor(\bt{T},\bt{F})
				= H_\lor(\bt{F},\bt{T}) = \bt{T}\,, \\
			H_\lor(\bt{F},\bt{F}) &= \bt{F}\,.
		\end{align*}
		\item the \df{if ... then} truth function:
		\begin{align*}
			H_\to(\bt{T},\bt{T}) &= H_\to(\bt{F},\bt{T})
				= H_\to(\bt{F},\bt{F}) = \bt{T}\,, \\
			H_\to(\bt{T},\bt{F}) &= \bt{F}\,.
		\end{align*}
		\item the \df{if and only if} truth function:
		\begin{align*}
			H_\lra(\bt{T},\bt{T}) &= H_\lra(\bt{F},\bt{F})
				= \bt{T}\,, \\
			H_\lra(\bt{T},\bt{F}) &= H_\lra(\bt{F},\bt{T})
				= \bt{F}\,.
		\end{align*}
		\item the \df{not} truth function:
		\[
			H_\neg(\bt{T}) = \bt{F}\,,
			\quad 
			H_\neg(\bt{F}) = \bt{T}\,.
		\]
	\end{itemize}
\end{example}

\begin{definition}[Predicate]
	A subset of the set of $n$-tuples in $A$ is called an 
	\df{$n$-ary predicate in $A$}\index{logic!predicate}.
\end{definition}

If $P$ represents such a predicate, then $P(a_1, \cdots, a_n)$ 
means that the $n$-tuple $(a_1, \cdots, a_n)$ is in $P$. We say 
\df{unary} for 1-ary and \df{binary} for 2-ary. Note that a unary 
function from $A$ to $B$ is a mapping from $A$ to $B$, and that a 
unary predicate in $A$ is a subset of $A$.

%%%%%%%%%%%%%%%%%%%%%%%%%%%%%%%%%%%%%%%%%%%%%%%%%%%%%%%%%%%%%%%%%%
%%%%%%%%%%%%%%%%%%%%%%%%%%%%%%%%%%%%%%%%%%%%%%%%%%%%%%%%%%%%%%%%%%

\newpage

\section{First-Order Languages}

Main Idea: First-order languages belong to the syntactical study of 
axiom systems. They describe the symbols to be used.

\begin{definition}[First-order Language]
	A \df{first-order language}\index{logic!first-order language} 
	has as symbols the following:
	\begin{enumerate}
		\item[a)] the variables
		\[ x,y,z,w,x',y',z',w',x'', \dots \,; \]
		
		\item[b)] for each $n$, the $n$-ary function symbols and 
		the $n$-ary predicate symbols;
		
		\item[c)] the symbols $\neg$, $\lor$, and $\exists$.
	\end{enumerate}
\end{definition}

Remark: The equality symbol = must be among the binary predicate 
symbols.

The first-order language $L'$ is an 
\df{extension}\index{logic!extension of language} of the 
first-order language $L$ if every nonlogical symbol of $L$ is a 
nonlogical symbol of $L'$.

\begin{definition}[Designators]
	$ $
	\begin{itemize}
		\item We define the \df{terms}\index{logic!terms} of a 
		first-order language by the generalized inductive 
		definition:
		\begin{enumerate}
			\item[i)] a variable is a term;
				
			\item[ii)] if $\bt{u}_1, \cdots, \bt{u}_n$ are 
			terms and \bt{f} is $n$-ary, then 
			$\bt{f}\bt{u}_1 \cdots \bt{u}_n$ is a term.
		\end{enumerate}
		
		\item An \df{atomic formula}\index{logic!atomic formula} is 
		an expression of the form $\bt{p}\bt{a}_1 \cdots \bt{a}_n$ 
		where \bt{p} is $n$-ary. We define the 
		\df{formulas}\index{logic!formulas} of a first-order 
		language by the generalized inductive definition:
		\begin{enumerate}
			\item[i)] an atomic formula is a formula;
			
			\item[ii)] if \bt{u} is a formula, then $\neg \bt{u}$ 
			is a formula;
			
			\item[iii)] if \bt{u} and \bt{v} are formulas, then 
			$\lor \bt{u}\bt{v}$ is a formula;
			
			\item[iv)] if \bt{u} is a formula, then $\exists \bt{x} 
			\bt{u}$ is a formula.
		\end{enumerate}
		\item A \df{designator}\index{logic!designator} is an 
		expression which is either a term or a formula.
	\end{itemize}
\end{definition}

%%%%%%%%%%%%%%%%%%%%%%%%%%%%%%%%%%%%%%%%%%%%%%%%%%%%%%%%%%%%%%%%%%
%%%%%%%%%%%%%%%%%%%%%%%%%%%%%%%%%%%%%%%%%%%%%%%%%%%%%%%%%%%%%%%%%%

\newpage

\section{Structures}

Main Idea: Structures describe the semantics, the meaning, of 
first-order languages; We attach meanings to the symbols. 
Individuals of the structure are given names. Add names to $L$ to 
get $L(\ms{A})$. Assign every variable-free term of $L(\ms{A})$ to 
an individual of the structure. Closed formulas of $L(\ms{A})$ are 
then given truth values, specifying if they're true in the 
structure. Then can say \bt{A} is valid in $\ms{A}$ if every 
instantiation of \bt{A} is true in $\ms{A}$. A model is a structure 
for a theory of a language in which all the nonlogical axioms of 
the theory are valid.

%%%%%%%%%%%%%%%%%%%%%%%%%%%%%%%%%%%%%%%%%%%%%%%%%%%%%%%%%%%%%%%%%%

\subsection{Definition}

\begin{definition}[Structure]
	Let $L$ be a first-order language. A 
	\df{structure}\index{logic!structure} $\ms{A}$ for $L$ consists 
	of the following things:
	\begin{enumerate}
		\item[i)] A nonempty set $| \ms{A} |$, called the 
		\df{universe} of $\ms{A}$. The elements of $| \ms{A} |$ are 
		called the \df{individuals} of $\ms{A}$.
		
		\item[ii)] For each $n$-ary function symbol \bt{f} of $L$, 
		an $n$-ary function $\bt{f}_\ms{A}$ from $| \ms{A} |$ to $| 
		\ms{A} |$.
		
		\item[iii)] For each $n$-ary predicate symbol \bt{p} of $L$ 
		other than =, an $n$-ary predicate $\bt{p}_\ms{A}$ in $| 
		\ms{A} |$.
	\end{enumerate}
\end{definition}

For each individual $a$ of $\ms{A}$, we choose a new constant, 
called the \df{name}\index{logic!name} of $a$. The first-order 
language obtained from $L$ by adding all the names of individuals 
of $\ms{A}$ is designated by $L(\ms{A})$. A formula \bt{A} is 
\df{closed}\index{logic!closed formula} if no variable is free in 
\bt{A}.

%%%%%%%%%%%%%%%%%%%%%%%%%%%%%%%%%%%%%%%%%%%%%%%%%%%%%%%%%%%%%%%%%%

\subsection{Individuals and Truth of Formulas in $\ms{A}$}

For each variable-free term \bt{a} of $L(\ms{A})$:
\begin{itemize}
	\item If \bt{a} is a name, $\ms{A}(\bt{a})$ is the individual 
	of which \bt{a} is the name.
	
	\item If \bt{a} is $\bt{f}\bt{a}_1 \cdots \bt{a}_n$, 
	$\ms{A}(\bt{a})$ is $\bt{f}_\ms{A} \left( \ms{A}(\bt{a}_1), 
	\cdots, \ms{A}(\bt{a}_n) \right)$.
\end{itemize}

\noindent
For each closed formula \bt{A} of $L(\ms{A})$:
\begin{itemize}
	\item If \bt{A} is $\bt{a} = \bt{b}$, $\ms{A}(\bt{A}) = \bt{T}$ 
	if and only if $\ms{A}(\bt{a}) = \ms{A}(\bt{b})$.
	
	\item If \bt{A} is $\bt{p}\bt{a}_1 \cdots \bt{a}_n$, where 
	\bt{p} is not =, $\ms{A}(\bt{A}) = \bt{T}$ if and only if 
	$\bt{p}_\ms{A} \left( \ms{A}(\bt{a}_1), \cdots, 
	\ms{A}(\bt{a}_n) \right)$.
	
	\item If \bt{A} is $\neg \bt{B}$, $\ms{A}(\bt{A})$ is 
	$\Hneg{}(\ms{A}(\bt{B}))$.
	
	\item If \bt{A} is $\lor \bt{B} \bt{C}$, $\ms{A}(\bt{A})$ 
	is $\Hor{}(\ms{A}(\bt{B}),\ms{A}(\bt{C}))$.
	
	\item If \bt{A} is $\existsx{}\bt{B}$, $\ms{A}(\bt{A}) = 
	\bt{T}$ if and 
	only if $\ms{A}(\Bxi{}) = \bt{T}$ for some \bt{i} in 
	$L(\ms{A})$.
\end{itemize}

If \bt{A} is a formula of $L$, an $\ms{A}$-instance of \bt{A} is a 
closed formula of the form $\bt{A}[\bt{i}_1, \dots, \bt{i}_n]$ in 
$L(\ms{A})$.
A formula \bt{A} is \df{valid}\index{logic!valid formula} in 
$\ms{A}$ if $\ms{A}(\bt{A}') = \bt{T}$ for every $\ms{A}$-instance 
$\bt{A}'$ of \bt{A}.
A formula is \df{elementary}\index{logic!elementary formula}  if it 
is either an atomic formula or an instantiation.

%%%%%%%%%%%%%%%%%%%%%%%%%%%%%%%%%%%%%%%%%%%%%%%%%%%%%%%%%%%%%%%%%%
%%%%%%%%%%%%%%%%%%%%%%%%%%%%%%%%%%%%%%%%%%%%%%%%%%%%%%%%%%%%%%%%%%

\newpage

\section{First-Order Theories}

Main Idea: The primary object of a formal system is to provide a 
framework for proving theorems.

%%%%%%%%%%%%%%%%%%%%%%%%%%%%%%%%%%%%%%%%%%%%%%%%%%%%%%%%%%%%%%%%%%

\subsection{Definition}

\begin{definition}[Formal System]
	A \df{formal system} $F$ consists of a language $L(F)$, axioms 
	of $F$, and rules of inference of $F$ which enable us to 
	conclude theorems from the axioms.
\end{definition}

\begin{definition}[First-Order Theory]
	A \df{first-order theory}\index{logic!first-order theory}, or 
	simply a \df{theory}, is a formal system $T$ such that
	\begin{enumerate}
		\item[i)] the language of $T$ is a first-order language;
		
		\item[ii)] the axioms of $T$ are the logical axioms of 
		$L(T)$ and certain further axioms called the \df{nonlogical 
		axioms};
		
		\item[iii)] the rules of $T$ are the expansion rule, the 
		contraction rule, the associative rule, the cut rule, and 
		the $\exists$-introduction rule.
	\end{enumerate}
\end{definition}

The logical axioms\index{logic!logical axioms} are the following:
\begin{itemize}
	\item Propositional axiom: $\propax$
	
	\item Substitution axiom: $\subax$
	
	\item Identity axiom: $\idax$
	
	\item Equality axiom: 
	\[ \eqax \] or \[ \eqpredax \]
\end{itemize}

The rules are the following:
\begin{itemize}
	\item Expansion: Infer $\bt{B} \lor \bt{A}$ from \bt{A}.
	
	\item Contraction: Infer \bt{A} from $\bt{A} \lor \bt{A}$.
	
	\item Associative: Infer $(\bt{A} \lor \bt{B}) \lor \bt{C}$ 
	from $\bt{A} \lor (\bt{B} \lor \bt{C})$.
	
	\item Cut: Infer $\bt{B} \lor \bt{C}$ from $\bt{A} \lor \bt{B}$ 
	and $\neg \bt{A} \lor \bt{C}$.
	
	\item $\exists$-Introduction: If \bt{x} is not free in \bt{B}, 
	infer $\existsx \bt{A} \ra \bt{B}$ from $\bt{A} \ra \bt{B}$.
\end{itemize}

%%%%%%%%%%%%%%%%%%%%%%%%%%%%%%%%%%%%%%%%%%%%%%%%%%%%%%%%%%%%%%%%%%

\subsection{Models of Theories}

\begin{definition}[Model]
	A \df{model}\index{logic!model} of a theory $T$ is a structure 
	for $L(T)$ in which all the nonlogical axioms of $T$ are valid. 
	A formula is \df{valid in $T$} if it is valid in every model of 
	$T$; equivalently, if it is a logical consequence of the 
	nonlogical axioms of $T$.
\end{definition}

\begin{example}[Elementary Theory of Groups, $G$]
	The only nonlogical symbol of $G$ is the binary function symbol 
	$\cdot$. The nonlogical axioms of $G$ are:
	\begin{enumerate}
		\item[G1.] $(x \cdot y) \cdot z = x \cdot (y \cdot z)$\,.
		
		\item[G2.] $\exists x (\forall y(x \cdot y = y) \land 
		\forall y \exists z (z \cdot y = x))$\,.
	\end{enumerate}
	
	A model of $G$ would be the set of invertible $2 \times 2$ 
	matrices under matrix multiplication.
\end{example}

%%%%%%%%%%%%%%%%%%%%%%%%%%%%%%%%%%%%%%%%%%%%%%%%%%%%%%%%%%%%%%%%%%

\subsection{Definitions Relevant to Theories}

\begin{itemize}
	\item A theory $T'$ is an \df{extension}\index{logic!extension 
	of theory} of a theory $T$ if $L(T')$ is an extension of $L(T)$ 
	and every theorem of $T$ is a theorem of $T'$. A 
	\df{conservative extension} of $T$ is an extension $T'$ of $T$ 
	such that every formula of $T$ which is a theorem of $T'$ is 
	also a theorem of $T$. The theories	$T$ and $T'$ are 
	\df{equivalent}\index{logic!equivalent theories} if each is an 
	extension of the other, i.e., they have the same language and 
	the same theorems.
	
	\item A theory $T$ is 
	\df{inconsistent}\index{logic!inconsistent} if every formula of 
	$T$ is a theorem of $T$; otherwise $T$ is 
	\df{consistent}\index{logic!consistent}
	
	\item A formula \bt{A} of $T$ is 
	\df{undecidable}\index{logic!undecidable} in $T$ if neither 
	\bt{A} nor $\neg \bt{A}$ is a theorem of $T$; otherwise \bt{A} 
	is \df{decidable}\index{logic!decidable} in $T$.
	
	\item A theory is \df{complete}\index{logic!complete} if it is 
	consistent and if every closed formula in $T$ is decidable in 
	$T$.
	
	\item A theory is \df{open}\index{logic!open theory} if all of 
	its nonlogical axioms are open (do not contain quantifiers).
\end{itemize}

%%%%%%%%%%%%%%%%%%%%%%%%%%%%%%%%%%%%%%%%%%%%%%%%%%%%%%%%%%%%%%%%%%

\subsection{Truth of Formulas of a Theory}

\begin{definition}[Truth Valuation]
	A \df{truth valuation}\index{logic!truth valuation for a 
	theory}	for $T$ is a mapping from the set of elementary 
	formulas in $T$ to the set of truth values.
\end{definition}

We define a truth value $V(\bt{A})$ for every formula \bt{A} by 
induction:
\begin{itemize}
	\item If \bt{A} is elementary, then $V(\bt{A})$ is already 
	defined;
	
	\item If \bt{A} is $\neg \bt{B}$, then $V(\bt{A}) = 
	\Hneg{}(V(\bt{B}))$;
	
	\item If \bt{A} is $\lor \bt{B} \bt{C}$, then $V(\bt{A}) = 
	\Hor{}(V(\bt{B}),V(\bt{C}))$.
\end{itemize}

%%%%%%%%%%%%%%%%%%%%%%%%%%%%%%%%%%%%%%%%%%%%%%%%%%%%%%%%%%%%%%%%%%

\subsection{Theorems in First-Order Theories}

\begin{theorem}[Validity Theorem]
	If $T$ is a theory, then every theorem of $T$ is valid in $T$.
\end{theorem}

%%%%%%%%%%%%%%%%%%%%%%%%%%%%%%%%%%%%%%%%%%%%%%%%%%%%%%%%%%%%%%%%%%
%%%%%%%%%%%%%%%%%%%%%%%%%%%%%%%%%%%%%%%%%%%%%%%%%%%%%%%%%%%%%%%%%%

\newpage

\section{The Characterization Problem}

\begin{definition}[Characterization Problem]
	The \df{characterization problem}\index{logic!characterization 
	problem} for a formal system $F$ is the following: find a 
	necessary and sufficient condition that a formula of $F$ be a 
	theorem of $F$.
\end{definition}

Remark: We consider the characterization problem for theories.

%%%%%%%%%%%%%%%%%%%%%%%%%%%%%%%%%%%%%%%%%%%%%%%%%%%%%%%%%%%%%%%%%%

\subsection{The Reduction Theorem}

Main Idea: To solve the characterization problem for all theories, 
it suffices to solve it for theories with no nonlogical axioms.

\begin{theorem}[Reduction Theorem]
	Let $\Gamma$ be a set of formulas in the theory $T$, and let 
	$A$ be a formula of $T$. Then \bt{A} is a theorem of 
	$T[\Gamma]$ iff there is a theorem of $T$ of the form $\bt{B}_1 
	\ra \dots \ra \bt{B}_n \ra \bt{A}$, where each $\bt{B}_i$ is 
	the closure of a formula in $\Gamma$.
\end{theorem}

\begin{theorem}[Reduction Theorem for Consistency]
	Let $\Gamma$ be a nonempty set of formulas in the theory $T$. 
	Then $T[\Gamma]$ is inconsistent iff there is a theorem of $T$ 
	which is a disjunction of negations of closures of distinct 
	formulas in $\Gamma$.
\end{theorem}

\begin{corollary}
	Let $\bt{A}'$ be the closure of \bt{A}. Then \bt{A} is a 
	theorem of $Y$ iff $T[\neg \bt{A}']$ is inconsistent.
\end{corollary}

%%%%%%%%%%%%%%%%%%%%%%%%%%%%%%%%%%%%%%%%%%%%%%%%%%%%%%%%%%%%%%%%%%

\subsection{Solutions to the Characterization Problem}

A trivial solution is that a formula is a theorem if and only if it 
has a proof.

\subsubsection{The Completeness Theorem}\index{logic!completeness 
theorem}

Main Idea: The completeness theorem, in either form, establishes an 
equivalence between a syntactical concept and a semantical concept. 
That is, it deals with concrete and abstract objects.

\begin{theorem}[Completeness Theorem, First Form (G\"odel)]
	A formula \bt{A} of a theory $T$ is a theorem of $T$ iff it is 
	valid in $T$.
\end{theorem}

\begin{theorem}[Completeness Theorem, Second Form]
	A theory $T$ is consistent iff it has a model.
\end{theorem}

The second form is proved using the concept of Henkin theories and 
special constants. It is outlined below.

\begin{definition}
	A Henkin theory\index{logic!Henkin theory} is a theory such 
	that each closed instantiation $\existsx \bt{A}$ of $T$, there 
	is a constant \bt{e} such that \\ $\vdash_T \existsx \bt{A} \ra 
	\bt{A}_\bt{x}[\bt{e}]$.
\end{definition}

One can extend any consistent theory to a Henkin theory by adding a 
special constant\index{logic!special constant} $\bt{r} = c_{\existsx
\bt{A}}$ for every case where $\existsx \bt{A}$ is a theorem, yet 
there is no such \bt{e}. If \bt{r} is the special constant for 
$\existsx \bt{A}$, then the formula $\existsx \bt{A} \ra 
\bt{A}_\bt{x}[\bt{r}]$ is called the \df{special axiom for} 
\bt{r}\index{logic!special axiom}. One can then extend this theory 
to a complete Henkin theory, which one can prove has a model. After 
showing that a theory has a model if its extension has a model, one 
can conclude that the original theory has a model.

%%%%%%%%%%%%%%%%%%%%%%%%%%%%%%%%%%%%%%%%%%%%%%%%%%%%%%%%%%%%%%%%%%

\subsubsection{The Consistency Theorem}

Main Idea: The consistency theorem deals only with concrete objects 
and is therefore finitary. Say we have a 
model of an open theory. The completeness theorem gives a proof of 
the consistency of the theory. The consistency theorem then allows 
us to convert this into a finitary proof of the consistency of the 
theory.

We start with some necessary definitions.

\begin{itemize}
	\item A theory is \df{open}\index{logic!open theory} if all of 
	its nonlogical axioms are open (do not contain quantifiers).
	
	\item A formula is a 
	\df{quasi-tautology}\index{logic!quasi-tautolgy} if it is a 
	tautological consequence of instances of identity axioms and 
	equality axioms.
\end{itemize}

\begin{theorem}[The Consistency Theorem (Hilbert-Ackermann)]
	An open theory $T$ is inconsistent if and only if there is a 
	quasi-tautology which is a disjunction of negations of 
	instances of nonlogical axioms of $T$.
\end{theorem}

%%%%%%%%%%%%%%%%%%%%%%%%%%%%%%%%%%%%%%%%%%%%%%%%%%%%%%%%%%%%%%%%%%

\subsubsection{Herbrand's Theorem}

Main Idea: Herbrand's theorem is a finitary solution of the 
characterization problem for all theories.

\begin{theorem}[Herbrand's Theorem]
	Let $T$ be a theory with no nonlogical axioms, and let \bt{A} 
	be a closed formula in prenex form in $T$. Then \bt{A} is a 
	theorem of $T$ if and only if there is a quasi-tautology which 
	is a disjunction of instances of the matrix of $\bt{A}_H$.
\end{theorem}

%%%%%%%%%%%%%%%%%%%%%%%%%%%%%%%%%%%%%%%%%%%%%%%%%%%%%%%%%%%%%%%%%%

\section{Interpretations}






































