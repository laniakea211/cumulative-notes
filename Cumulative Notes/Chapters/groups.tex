\chapter{Group Theory}

Abstract algebra is, as its name suggests, extremely abstract. It 
arose as an attempt to abstract the techniques used in algebraic 
equations, number theory, and geometry.
There are many specific examples of groups, and the power of the 
abstract point of view becomes apparent when results for all of 
these examples are obtained by providing a single result for the 
abstract group. The information in this chapter is based on Dummit 
and Foote's \emph{Abstract Algebra} (2004).

%%%%%%%%%%%%%%%%%%%%%%%%%%%%%%%%%%%%%%%%%%%%%%%%%%%%%%%%%%%%%%%%%%

\newpage

\section{Notation}

\begin{itemize}
	\item We write $ab$ for $a \cdot b$.
	
	\item We denote the identity element of an abstract group $(G, 
	\cdot)$ by 1.
	
	\item We denote $xx \dots x$ ($n$ terms) by $x^n$ and $x^{-1} 
	x^{-1} \dots x^{-1}$ ($n$ terms) by $x^{-n}$.
	
	\item When the operation is +, the identity will be denoted by 
	0, and for any element $a$, the inverse will be written $-a$. 
	Moreover, $a+ \dots + a$ ($n > 0$ terms) will be written $na$; 
	$-a - a \dots -a$ ($n$ terms) will be written $-na$ and $0a = 
	0$.
	
	\item For each $n \in \zplus$ let $GL_n(F)$ be the set of all 
	$n \times n$ matrices whose entries come from $F$ and whose 
	determinant is nonzero.
\end{itemize}

%%%%%%%%%%%%%%%%%%%%%%%%%%%%%%%%%%%%%%%%%%%%%%%%%%%%%%%%%%%%%%%%%%

\newpage

\section{Definition and Examples}

\begin{definition}[Binary Operation]
	$ $
	\begin{enumerate}
		\item A \df{binary operation}\index{group theory!binary 
		operation} $\cdot$ on a set $G$ is a function $\cdot : G 
		\times G \to G$. For any $a,b \in G$ we	write $a \cdot b$.
		
		\item A binary operation $\cdot$ on a set $G$ is 
		\df{associative}\index{group theory!associative} if for all 
		$a,b,c \in G$ we have $a \cdot (b \cdot c) = (a \cdot b) 
		\cdot c$.
		
		\item If $\cdot$ is a binary operation on a set $G$ we say 
		elements $a$ and $b$ of $G$ \df{commute}\index{group 
			theory!commutative} if $a \cdot b = b \cdot a$. We say 
		$\cdot$ (or $G$) is \df{commutative} if for all $a,b \in 
		G$, $a \cdot b = b \cdot a$.
		
		\item Suppose that $\cdot$ is a binary operation on a set 
		$G$ and $H$ is a subset of $G$. If the restriction of 
		$\cdot$ to $H$ is a binary operation on $H$, i.e., for all 
		$a,b \in H$, $a \cdot b \in H$, then $H$ is said to be 
		\df{closed}\index{group theory!closed under operation} 
		under $\cdot$.
	\end{enumerate}
\end{definition}

\begin{definition}[Group]
	$ $
	\begin{enumerate}
		\item A \df{group}\index{group theory!group} is an ordered 
		pair $(G,\cdot)$ where $G$ is a set and $\cdot$ is a binary 
		operation on $G$ satisfying the following axioms:
		\begin{enumerate}
			\item $\cdot$ is associative,
			
			\item there exists an element $e$ in $G$, called an 
			\df{identity}\index{group theory!identity element} of 
			$G$, such that for all $a \in G$ we have $a \cdot e = e 
			\cdot a = a$,
			
			\item for each $a \in G$ there is an element $a^{-1}$ 
			of $G$, called an \df{inverse}\index{group 
				theory!inverse of element} of $a$, such that $a 
				\cdot 
			a^{-1} = a^{-1} \cdot a = e$.
		\end{enumerate}
		
		\item The group $(G, \cdot)$ is called 
		\df{abelian}\index{group theory!abelian} (or 
		\df{commutative}\index{group theory!commutative group}) if 
		$a \cdot b = b \cdot a$ for all $a,b \in G$.
	\end{enumerate}
\end{definition}

Remark: We say $(G, \cdot)$ is a \df{finite group}\index{group 
theory!finite 
group} if in addition $G$ is a 
finite set.

\begin{example}
	$ $
	\begin{itemize}
		\item $\integers$, $\rationals$, $\reals$, and $\complex$ 
		are groups under + with $e = 0$ and $a^{-1} = -a$, for all 
		$a$.
		
		\item $\rationals \setminus \{ 0 \}$, $\reals \setminus \{ 
		0 \}$, $\complex \setminus \{ 0 \}$, $\rationals^+$, 
		$\reals^+$ are groups under $\times$ with $e = 1$ and 
		$a^{-1} = \frac{1}{a}$, for all $a$.
	\end{itemize}
\end{example}

%%%%%%%%%%%%%%%%%%%%%%%%%%%%%%%%%%%%%%%%%%%%%%%%%%%%%%%%%%%%%%%%%%

\newpage

\section{How to Form New Groups From Given Ones}

\begin{itemize}
	\item If $(A,\cdot)$ and $(B,*)$ are groups, we can form a new
	group $(A \times B, \diamond)$, called their \df{direct 
	product}\index{group theory!direct product}, whose elements are 
	those in the Cartesian product $A \times B$ and whose operation 
	is defined componentwise:
	\[
		(a_1,b_1)(a_2,b_2) = (a_1 \cdot a_2, b_1 * b_2)\,.
	\]

	\begin{example}
		If we take $A = B = \reals$ (both operations addition), 
		$\reals \times \reals$ is the familiar Euclidean plane.
	\end{example}

	\item Let $G$ be a group. The subset $H$ of $G$ is a 
	\df{subgroup}\index{group theory!subgroup} of $G$ if $H$ is 
	nonempty and $H$ is closed under products and inverses. If $H$ 
	is a subgroup of $G$ we shall write $H \leq G$.
	
	Remark: Subgroups of $G$ are just subsets of $G$ which are 
	themselves groups with respect to the operation defined in $G$.

\end{itemize}

%%%%%%%%%%%%%%%%%%%%%%%%%%%%%%%%%%%%%%%%%%%%%%%%%%%%%%%%%%%%%%%%%%

\newpage

\section{Basic Theorems About Groups}

\begin{theorem}[Uniqueness of Identity and Inverse]
	If $(G, \cdot)$ is a group, then
	\begin{enumerate}
		\item the identity of $(G, \cdot)$ is unique
		
		\item for each $a \in G$, $a^{-1}$ is uniquely determined
		
		\item $(a^{-1})^{-1} = a$ for all $a \in G$
		
		\item $(a \cdot b)^{-1} = (b^{-1}) \cdot (a^{-1})$
		
		\item for any $a_1, a_2, \dots, a_n \in G$ the value of 
		$a_1 \cdot a_2 \cdot \dots \cdot a_n$ is independent of how 
		the expression is bracketed (this is called the 
		\df{generalized associative law}).
	\end{enumerate}
\end{theorem}

\begin{theorem}[Cancellation Laws]
	Let $(G, \cdot)$ be a group and let $a,b \in G$. The equations 
	$ax = b$ 
	and $ya = b$ have unique solutions for $x,y \in G$. In 
	particular, the left and right cancellation laws hold in $G$, 
	i.e.,
	\begin{enumerate}
		\item if $au = av$, then $u=v$, and
		
		\item if $ub=vb$, then $u=v$.
	\end{enumerate}
\end{theorem}

%%%%%%%%%%%%%%%%%%%%%%%%%%%%%%%%%%%%%%%%%%%%%%%%%%%%%%%%%%%%%%%%%%

\newpage

\section{Order of An Element of A Group}

\begin{definition}
	For $G$ a group and $x \in G$, the \df{order}\index{group 
	theory!order of element} of $x$ is the 
	smallest positive integer $n$ such that $x^n = 1$. We denote 
	this integer by $|x|$. In this case $x$ is said to be of order 
	$n$. If no positive power of $x$ is the identity, the order of 
	$x$ is defined to be infinity and $x$ is said to be of infinite 
	order.
\end{definition}

Remark: The order of an element in a group is the same as the 
cardinality of the set of all its distinct powers, which is why we 
use the same notation as cardinality.

\begin{example}
	$ $
	\begin{itemize}
		\item An element of a group has order 1 if and only if it 
		is the identity.
		
		\item In the multiplicative groups $\reals \setminus \{ 0 
		\}$ or $\rationals \setminus \{ 0 \}$ the element $-1$ has 
		order 2 and all other nonidentity elements have infinite 
		order.
	\end{itemize}
\end{example}

%%%%%%%%%%%%%%%%%%%%%%%%%%%%%%%%%%%%%%%%%%%%%%%%%%%%%%%%%%%%%%%%%%

\newpage

\section{Multiplication Table}

\begin{definition}[Multiplication/Group Table]
	Let $G = \{ g_1, g_2, \dots, g_n \}$ be a finite group with 
	$g_1 = 1$. The \df{multiplication table}\index{group 
	theory!multiplication/group table} or \df{group table} of 
	$G$ is the $n \times n$ matrix whose $i, j$ entry is the group 
	element $g_ig_j$.
\end{definition}

Remark: ``For a finite group the multiplication table contains, in 
some sense, all the information about the group. Computationally, 
however, it is an unwieldy object (being of size the square of the 
group order) and visually it is not a very useful object for 
determining properties of the group. ... Part of our initial 
development of the theory of groups (finite groups in particular) 
is directed towards a more conceptual way of visualizing the 
internal structure of groups.''

%%%%%%%%%%%%%%%%%%%%%%%%%%%%%%%%%%%%%%%%%%%%%%%%%%%%%%%%%%%%%%%%%%

\newpage 

\section{Generators and Relations}\label{generators and relations}

\begin{definition}[Generators]
	A subset $S$ of elements of a group $G$ with the property that 
	every element of $G$ can be written as a (finite) product of 
	elements of $S$ and their inverses is called a set of 
	\df{generators}\index{group theory!generators} of $G$. We 
	write $G = \langle S \rangle$ and say $G$ is generated by $S$ 
	or $S$ generates $G$.
\end{definition}

\begin{example}
	$ $
	\begin{itemize}
		\item The integer 1 is a generator for the additive group 
		$\integers$ of integers since every integer is a sum of a 
		finite number of $+1$'s and $-1$'s, so $\integers = \langle 
		1 \rangle$.
		
		\item As defined in Section \ref{dihedral groups}, $D_{2n} 
		= \langle r, s \rangle$.
	\end{itemize}
\end{example}

\begin{definition}
	Any equations in a general group $G$ that the generators 
	satisfy are called the \df{relations}\index{group 
	theory!relations} in $G$.
\end{definition}

\begin{example}
	In $D_{2n}$ we have the relations $r^n = 1$, $s^2 = 1$, and $rs 
	= sr^{-1}$.
\end{example}

If some group $G$ is generated by a subset $S$ and there is some 
collection of relations, say $R_1, R_2, \dots, R_m$ (here each 
$R_i$ is an equation in the elements from $S \cup \{ 1 \}$) such 
that any relation among the elements of $S$ can be deduced from 
these, we shall call these generators and relations a 
\df{presentation}\index{group theory!presentation} of $G$ and write
\[
	G = \langle S \mid R_1, R_2, \dots, R_m \rangle\,.
\]

For example, $D_{2n} = \langle r,s \mid r^n = s^2 = 1, rs = sr^{-1} 
\rangle$ is one presentation of the dihedral group $D_{2n}$. 
Working in terms of presentations often simplifies the situation.

%%%%%%%%%%%%%%%%%%%%%%%%%%%%%%%%%%%%%%%%%%%%%%%%%%%%%%%%%%%%%%%%%%

\newpage

\section{Important Groups}

\subsection{Dihedral Groups}\label{dihedral groups}

Main Idea: An important family of examples of groups is the class 
of groups whose elements are symmetries of geometric objects.

We rigorously define the symmetries of a regular $n$-gon. Label 
each vertex with a distinct integer $i$ for $1 \leq i \leq n$. 
Each symmetry $s$ can be described uniquely by the 
corresponding permutation $\sigma$ of $\{1, 2, 3, \dots, n \}$ 
where if the symmetry $s$ puts vertex $i$ in the place where 
vertex $j$ was originally, then $\sigma$ is the permutation 
sending $i$ to $j$. There are $n$ rotation symmetries and $n$ 
reflection symmetries.

\begin{definition}[Dihedral Group]\index{group theory!dihedral 
group}
	The set of symmetries $s$ of a regular $n$-gon, together with 
	the binary operation of composition form a group called the 
	\df{dihedral group of order $2n$}.
\end{definition}

\subsubsection{Notation}

\begin{itemize}
	\item Let $r$ be the rotation clockwise about the origin 
	through $2\pi/n$ radian. 
	
	\item Let $s$ be the reflection about the line of symmetry 
	through vertex 1 and the origin.
\end{itemize}

\subsubsection{Theorems About Dihedral Groups}

\begin{theorem}
	$ $
	\begin{enumerate}
		\item $1, r, r^2, \dots, r^{n-1}$ are distinct and $r^n = 
		1$, so $|r| = n$.
		
		\item $|s| = 2$.
		
		\item $s \neq r^i$ for any $i$.
		
		\item $sr^i \neq sr^j$, for all $0 \leq i$, $j \leq n - 1$ 
		with $i \neq j$, so
		\[
			D_{2n} = \{ 1, r, r^2, \dots, r^{n-1}, s, sr, sr^2, 
			\dots, sr^{n-1} \}\,.
		\]
		
		\item $rs = sr^{-1}$. This shows in particular that $r$ and 
		$s$ do not commute so that $D_{2n}$ is non-abelian.
		
		\item $r^is = sr^{-i}$, for all $0 \leq i \leq n$. This 
		indicates how to commute $s$ with powers of $r$.
	\end{enumerate}
\end{theorem}

$r$ and $s$ are called \df{generators} for the dihedral group. (See 
Section \ref{generators and relations}.)

%%%%%%%%%%%%%%%%%%%%%%%%%%%%%%%%%%%%%%%%%%%%%%%%%%%%%%%%%%%%%%%%%%

\newpage

\subsection{Symmetric Groups}

\begin{definition}[Symmetric Group]\index{group theory!symmetric 
group}
	Let $\Omega$ be any nonempty set and let $S_\Omega$ be the set 
	of all bijections from $\Omega$ to itself (i.e., the set of all 
	permutations of $\Omega$). The set $S_\Omega$ is a group under 
	function composition ($\circ$) called the \df{symmetric group 
	on the set $\Omega$}.
\end{definition}

Remark: In the special case when $\Omega = \{ 1, 2, 3, \dots, n 
\}$, the symmetric group on $\Omega$ is denoted $S_n$, the 
\df{symmetric group of degree $n$}.

\subsubsection{Cycle Decomposition}\index{group theory!cycle 
decomposition}

\begin{definition}[Cycle]\index{group theory!cycle}
	A \df{cycle} is a string of integers which represents the 
	elements of $S_n$ which cyclically permutes these integers (and 
	fixes all other integers). The cycle $(a_1 \, a_2 \, \dots \, 
	a_m)$ is the permutation which sends $a_i$ to $a_{i+1}$, $1 
	\leq i \leq m-1$ and sends $a_m$ to $a_1$.
\end{definition}

\begin{example}
	The cycle $(2 \, 1 \, 3)$ is the permutation which maps 2 to 1, 
	1 to 3, and 3 to 2.
\end{example}

Remark: Disjoint cycles commute.

For each $\sigma \in S_n$ the numbers from 1 to $n$ will 
be rearranged and grouped into $k$ cycles of the form
\[
(a_1 \, a_2 \, \dots \, a_{m_1})
(a_{m_1+1} \, a_{m_1+2} \, \dots \, a_{m_2})
\dots 
(a_{m_{k-1}+1} \, a_{m_{k-1}+2} \, \dots \, a_{m_k})
\]
from which the action of $\sigma$ on any number from 1 to $n$ 
can easily be read. The product of these cycles is called the 
\df{cycle decomposition} of $\sigma$.

\subsubsection{Theorems About Symmetric Groups}

\begin{itemize}
	\item The order of $S_n$ is $n!$.
	
	\item $S_n$ is a non-abelian group for all $n \geq 3$.
	
	\item The cycle decomposition of each permutation is the unique 
	way of expressing a permutation as a product of disjoint cycles 
	(up to rearranging its cycles and cyclically permuting the 
	numbers within each cycle).
	
	\item The order of a permutation is the l.c.m. of the lengths 
	of the cycles in its cycle decomposition.
\end{itemize}

%%%%%%%%%%%%%%%%%%%%%%%%%%%%%%%%%%%%%%%%%%%%%%%%%%%%%%%%%%%%%%%%%%

\newpage

\subsection{Matrix Groups}

\begin{definition}[General Linear Group]\index{group theory!general 
linear group}
	$GL_n(F)$, that is the set of all $n \times n$ matrices whose 
	entries come from a field $F$ and whose determinant is nonzero, 
	form a group under matrix multiplication called the \df{general 
	linear group of degree $n$}.
\end{definition}

\subsubsection{Theorems About Matrix Groups}

\begin{itemize}
	\item If $F$ is a field and $|F| < \infty$, then $|F| = p^m$ 
	for some prime $p$ and integer $m$.
	
	\item If $|F| = q < \infty$, then $|GL_n(F)| = (q^n - 1)(q^n - 
	q)(q^n - q^2) \dots (q^n - q^{n-1})$.
\end{itemize}

%%%%%%%%%%%%%%%%%%%%%%%%%%%%%%%%%%%%%%%%%%%%%%%%%%%%%%%%%%%%%%%%%%

\newpage

\subsection{The Quaternion Group}

\begin{definition}[Quaternion Group]\index{group theory!quaternion 
group}
	The \df{quaternion group}, $Q_8$, is defined by 
	\[
		\{ 1, -1, i, -i, j, -j, k, -k \}
	\]
	with product $\cdot$ computed as follows:
	\begin{gather}
		1 \cdot a = a \cdot 1 = a\,, \quad
			\text{for all } a \in Q_8 \\
		(-1) \cdot (-1) = 1\,, \quad (-1) \cdot a = a \cdot (-1)
			= -a\,, \quad \text{for all } a \in Q_8 \\
		i \cdot i = j \cdot j = k \cdot k = -1 \\
		i \cdot j = k\,, \quad j \cdot i = -k \\
		j \cdot k = i\,, \quad k \cdot j = -i \\
		k \cdot i = j\,, \quad i \cdot k = -j\,.
	\end{gather}
\end{definition}

Remark: $Q_8$ is a non-abelian group of order 8.

\subsubsection{Theorems About the Quaternion Group}

%%%%%%%%%%%%%%%%%%%%%%%%%%%%%%%%%%%%%%%%%%%%%%%%%%%%%%%%%%%%%%%%%%

\newpage

\section{Homomorphisms and Isomorphisms}

Main Idea: A map is a homomorphism if it respects the group 
structures of its domain and codomain. The notion of isomorphism 
makes precise the idea of when two groups ``look the same," that 
is, have exactly the same group-theoretic structure.

\begin{definition}[Homomorphism]\index{group theory!homomorphism}
	Let $(G, \cdot)$ and $(H, *)$ be groups. A map $\varphi:G \to 
	H$ 
	such that 
	\[
		\varphi(x \cdot y) = \varphi(x) * \varphi(y)\,, \quad 
		\text{for all } x,y \in G
	\]
	is called a \df{homomorphism}.
\end{definition}

\begin{definition}[Isomorphism]\index{group theory!isomorphism}
	The map $\varphi:G \to H$ is called an \df{isomorphism} and $G$ 
	and $H$ are said to be \df{isomorphic} or of the same 
	\df{isomorphism type}, written $G \cong H$, if 
	\begin{enumerate}
		\item $\varphi$ is a homomorphism, and
		\item $\varphi$ is a bijection.
	\end{enumerate}
\end{definition}

\subsubsection{Theorems About Isomorphisms}

\begin{itemize}
	\item The relation $\cong$ is an equivalence relation, the 
	equivalence classes of which are called \df{isomorphism 
	classes}.

	\item If $\varphi:G \to H$ is an isomorphism, then
	\begin{enumerate}
		\item $|G| = |H|$
		
		\item $G$ is abelian if and only if $H$ is abelian
		
		\item for all $x \in G$, $|x| = |\varphi(x)|$.
	\end{enumerate}

	\item Let $G$ be a finite group of order $n$ for which we have 
	a presentation and let $S = \{ s_1, \dots, s_m \}$ be the 
	generators. Let $H$ be another group and $\{ r_1, \dots, r_m 
	\}$ be the elements of $H$. Suppose that any relation satisfied 
	in $G$ by the $s_i$ is also satisfied in $H$ when each $s_i$ is 
	replaced by $r_i$. Then there is a unique homomorphism 
	$\varphi:G \to H$ which maps $s_i$ to $r_i$.
\end{itemize}

%%%%%%%%%%%%%%%%%%%%%%%%%%%%%%%%%%%%%%%%%%%%%%%%%%%%%%%%%%%%%%%%%%

\newpage

\section{Group Actions}

Main Idea: A group action is when a group acts on a set. 
Intuitively, a group action of $G$ on a set $A$ just means that 
every element $g$ in $G$ acts as a permutation on $A$ in a manner 
consistent with the group operations in $G$.

\begin{definition}[Group Action]\index{group theory!group action}
	A \df{group action} of a group $G$ on a set $A$ is a map from 
	$G \times A$ to $A$ (written as $g \cdot a$, for all $g \in G$ 
	and $a \in A$) satisfying the following properties:
	\begin{enumerate}
		\item $g_1 \cdot (g_2 \cdot a) = (g_1g_2) \cdot a$, for all 
		$g_1, g_2 \in G$, $a \in A$, and
		
		\item $1 \cdot a = a$, for all $a \in A$.
	\end{enumerate}
\end{definition}

For each fixed $g \in G$ we get a map $\sigma_g$ defined by 
\begin{gather}
	\sigma_g: A \to A \\
	\sigma_g(a) = g \cdot a\,,
\end{gather}
where $\sigma_g$ is a permutation of $A$ and the map from $G$ to 
$S_A$ defined by $g \mapsto \sigma_g$ is a homomorphism. This 
homomorphism is called the \df{permutation 
representation}\index{group theory!permutation representation of 
group action} associated to the given action.

If $G$ acts on a set $B$ and distinct elements of $G$ induce 
distinct permutations of $B$, the action is said to be 
\df{faithful}\index{group theory!faithful group action}.

The \df{kernel}\index{group theory!kernel of group action} of the 
action of $G$ on $B$ is defined to be $\{ g \in G \mid gb = b 
\text{ for all } b \in B \}$, namely the elements of $G$ which fix 
all the elements of $B$.

\begin{example}
	$ $
	\begin{itemize}
		\item For any nonempty set $A$ the symmetric group $S_A$ 
		acts on $A$ by $\sigma \cdot a = \sigma(a)$ for all $\sigma 
		\in S_A$, $a \in A$. The associated permutation 
		representation is the identity map from $S_A$ to itself.
		
		\item The axioms for a vector space $V$ over a field $F$ 
		include the two axioms that the multiplicative group 
		$F^\times$ act on the set $V$.
	\end{itemize}
\end{example}































