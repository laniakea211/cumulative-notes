\chapter{Solutions to D\&F's \emph{Abstract Algebra}}

\section*{Preliminaries}

\subsection{Basics}

In Exercises 1 to 4 let $\mc{A}$ be the set of $2 \times 2$ matrices with real number entries. Recall that matrix multiplication is defined by
\[
	\begin{pmatrix}
		a & b \\
		c & d
	\end{pmatrix}
	\begin{pmatrix}
		p & q \\
		r & s
	\end{pmatrix}
	=
	\begin{pmatrix}
		ap + br & aq + bs \\
		cp + dr & cq + ds
	\end{pmatrix}\,.
\]
Let
\[
	M = 
	\begin{pmatrix}
		1 & 1 \\
		0 & 1
	\end{pmatrix}
\]
and let
\[
	\mc{B} = \{ X \in \mc{A} \mid MX = XM \}\,.
\]

\begin{exercise}
	Determine which of the following elements of $\mc{A}$ lie in $\mc{B}$:
	\[
		\begin{pmatrix}
			1 & 1 \\
			0 & 1
		\end{pmatrix},
		\quad
		\begin{pmatrix}
			1 & 1 \\
			1 & 1
		\end{pmatrix},
		\quad
		\begin{pmatrix}
			0 & 0 \\
			0 & 0
		\end{pmatrix},
		\quad
		\begin{pmatrix}
			1 & 1 \\
			1 & 0
		\end{pmatrix},
		\quad
		\begin{pmatrix}
			1 & 0 \\
			0 & 1
		\end{pmatrix},
		\quad
		\begin{pmatrix}
			0 & 1 \\
			1 & 0
		\end{pmatrix}\,.
	\]
\end{exercise}

\begin{proof}[Solution]
	We have
	\begin{align*}
		\begin{pmatrix}
			1 & 1 \\
			0 & 1
		\end{pmatrix}
		\begin{pmatrix}
			1 & 1 \\
			0 & 1
		\end{pmatrix}
		&=
		\begin{pmatrix}
			1 & 1 \\
			0 & 1
		\end{pmatrix}
		\begin{pmatrix}
			1 & 1 \\
			0 & 1
		\end{pmatrix}\,,
		\text{ so }	
		\begin{pmatrix}
			1 & 1 \\
			0 & 1
		\end{pmatrix}
		\in \mc{B}\,;
		\\
		\begin{pmatrix}
			1 & 1 \\
			0 & 1
		\end{pmatrix}
		\begin{pmatrix}
			1 & 1 \\
			1 & 1
		\end{pmatrix}
		&=
		\begin{pmatrix}
			2 & 2 \\
			1 & 1
		\end{pmatrix}\,,
		\text{ and }
		\begin{pmatrix}
			1 & 1 \\
			1 & 1
		\end{pmatrix}
		\begin{pmatrix}
			1 & 1 \\
			0 & 1
		\end{pmatrix}
		=
		\begin{pmatrix}
			1 & 2 \\
			1 & 2
		\end{pmatrix}\,,
		\\
		&\text{ so }
		\begin{pmatrix}
			1 & 1 \\
			1 & 1
		\end{pmatrix}
		\notin \mc{B}\,;
		\\
		\begin{pmatrix}
			1 & 1 \\
			0 & 1
		\end{pmatrix}
		\begin{pmatrix}
			0 & 0 \\
			0 & 0
		\end{pmatrix}
		&=
		\begin{pmatrix}
			0 & 0 \\
			0 & 0
		\end{pmatrix}
		\begin{pmatrix}
			1 & 1 \\
			0 & 1
		\end{pmatrix}\,,
		\text{ so }
		\begin{pmatrix}
			0 & 0 \\
			0 & 0
		\end{pmatrix}
		\in \mc{B}\,;
		\\
		\begin{pmatrix}
			1 & 1 \\
			0 & 1
		\end{pmatrix}
		\begin{pmatrix}
			1 & 1 \\
			1 & 0
		\end{pmatrix}
		&=
		\begin{pmatrix}
			2 & 1 \\
			1 & 0
		\end{pmatrix}\,,
		\text{ and }
		\begin{pmatrix}
			1 & 1 \\
			1 & 0
		\end{pmatrix}
		\begin{pmatrix}
			1 & 1 \\
			0 & 1
		\end{pmatrix}
		=
		\begin{pmatrix}
			1 & 2 \\
			1 & 1
		\end{pmatrix}\,,
		\\
		&\text{ so }
		\begin{pmatrix}
			1 & 1 \\
			1 & 0
		\end{pmatrix}
		\notin \mc{B}\,;
		\\
		\begin{pmatrix}
			1 & 1 \\
			0 & 1
		\end{pmatrix}
		\begin{pmatrix}
			1 & 0 \\
			0 & 1
		\end{pmatrix}
		&=
		\begin{pmatrix}
			1 & 0 \\
			0 & 1
		\end{pmatrix}
		\begin{pmatrix}
			1 & 1 \\
			0 & 1
		\end{pmatrix}\,,
		\text{ so }	
		\begin{pmatrix}
			1 & 0 \\
			0 & 1
		\end{pmatrix}
		\in \mc{B}\,;
		\\
		\begin{pmatrix}
			1 & 1 \\
			0 & 1
		\end{pmatrix}
		\begin{pmatrix}
			0 & 1 \\
			1 & 0
		\end{pmatrix}
		&=
		\begin{pmatrix}
			1 & 1 \\
			1 & 0
		\end{pmatrix}\,,
		\text{ and }
		\begin{pmatrix}
			0 & 1 \\
			1 & 0
		\end{pmatrix}
		\begin{pmatrix}
			1 & 1 \\
			0 & 1
		\end{pmatrix}
		=
		\begin{pmatrix}
			0 & 1 \\
			1 & 1
		\end{pmatrix}\,,
		\\
		&\text{ so }
		\begin{pmatrix}
			0 & 1 \\
			1 & 0
		\end{pmatrix}
		\notin \mc{B}\,;
	\end{align*}
\end{proof}

%%%%%%%%%%%%%%%%%%%%%%%%%%%%%%%%%%%%%%%%%%%%%%%%%%%%%%%%%%%%%%%%%%

\begin{exercise}
	Prove that if $P, Q \in \mc{B}$, then $P + Q \in \mc{B}$ (where + denotes the usual sum of two matrices).
\end{exercise}

\begin{proof}
	We have
	\[
		M(P + Q) = MP + MQ = PM + QM = (P + Q)M\,.
	\]
\end{proof}

%%%%%%%%%%%%%%%%%%%%%%%%%%%%%%%%%%%%%%%%%%%%%%%%%%%%%%%%%%%%%%%%%%

\begin{exercise}
	Prove that if $P, Q \in \mc{B}$, then $P \cdot Q \in \mc{B}$ (where $\cdot$ denotes the usual product of two matrices).
\end{exercise}

\begin{proof}
	We have
	\[
		M(PQ) = (MP)Q = (PM)Q = P(MQ) = P(QM) = (PQ)M\,.
	\]
\end{proof}

%%%%%%%%%%%%%%%%%%%%%%%%%%%%%%%%%%%%%%%%%%%%%%%%%%%%%%%%%%%%%%%%%%

\begin{exercise}
	Find conditions on $p, q, r, s$ which determine precisely when
	$\begin{pmatrix}
		p & q \\
		r & s
	\end{pmatrix}
	\in \mc{B}$.
\end{exercise}

\begin{proof}[Solution]
	We have
	\[
		\begin{pmatrix}
			1 & 1 \\
			0 & 1
		\end{pmatrix}
		\begin{pmatrix}
			p & q \\
			r & s
		\end{pmatrix}
		=
		\begin{pmatrix}
			p + r & q + s \\
			r & s
		\end{pmatrix}\,,
	\]
	and
	\[
		\begin{pmatrix}
			p & q \\
			r & s
		\end{pmatrix}
		\begin{pmatrix}
			1 & 1 \\
			0 & 1
		\end{pmatrix}
		=
		\begin{pmatrix}
			p & p + q \\
			r & r + s
		\end{pmatrix}\,.
	\]
	Setting these resultant matrices equal to each other yields the required condition: $r = 0$ and $p = s$.
\end{proof}

%%%%%%%%%%%%%%%%%%%%%%%%%%%%%%%%%%%%%%%%%%%%%%%%%%%%%%%%%%%%%%%%%%

\begin{exercise}
	Determine whether the following functions $f$ are well defined:
	\begin{enumerate}
		\item[(a)] $f : \rationals \to \integers$ defined by $f(a/b)
			= a$.
		\item[(b)] $f : \rationals \to \rationals$ defined by
			$f(a/b) = a^2/b^2$.
	\end{enumerate}
\end{exercise}

\begin{proof}[Solution]
	\begin{enumerate}
		\item[(a)] This function $f$ is not well defined for the
			following two reasons: $f(1/2) \neq f(2/4)$ and $f(-1/2) \neq f(1/-2)$.
			
		\item[(b)] This function $f$ is well defined, since any
			negatives will be squared away and equivalent ways of writing a fraction give the same result: $f(na/nb) = n^2a^2/n^2b^2 = a^2/b^2 = f(a/b)$.
	\end{enumerate}
\end{proof}

%%%%%%%%%%%%%%%%%%%%%%%%%%%%%%%%%%%%%%%%%%%%%%%%%%%%%%%%%%%%%%%%%%

\begin{exercise}
	Determine whether the function $f : \rplus \to \integers$ defined by mapping a real number $r$ to the first digit to the right of the decimal point in a decimal expansion of $r$ is well defined.
\end{exercise}

\begin{proof}[Solution]
	This function $f$ is not well defined since $f(.\bar{9}) = 9$ 
	and $f(1) = 0$ even though $.\bar{9} = 1$.
\end{proof}

%%%%%%%%%%%%%%%%%%%%%%%%%%%%%%%%%%%%%%%%%%%%%%%%%%%%%%%%%%%%%%%%%%

\begin{exercise}
	Let $f : A \to B$ be a surjective map of sets. Prove that the relation
	\[
		a \sim b \text{ if and only if } f(a) = f(b)
	\]
	is an equivalence relation whose equivalence classes are the fibers of $f$.
\end{exercise}

\begin{proof}
	We first prove $\sim$ is reflexive. Since $f$ is well defined, $f(a) = f(a)$. Next we prove $\sim$ is symmetric. If $f(a) = f(b)$, then $f(b) = f(a)$. Finally, we show $\sim$ is transitive. If $f(a) = f(b)$ and $f(b) = f(c)$, then $f(a) = f(c)$.
	
	We now need to show that the equivalence classes of $\sim$ are the fibers of $f$. $a \sim b$ means $f(a) = f(b)$. This means that $a$ and $b$ map to the same element of $B$; that is, they belong to the same fiber of $f$.
\end{proof}

%%%%%%%%%%%%%%%%%%%%%%%%%%%%%%%%%%%%%%%%%%%%%%%%%%%%%%%%%%%%%%%%%%
%%%%%%%%%%%%%%%%%%%%%%%%%%%%%%%%%%%%%%%%%%%%%%%%%%%%%%%%%%%%%%%%%%

\subsection{Properties of the Integers}

\begin{exercise}
	For each of the following pairs of integers $a$ and $b$, determine their greatest common divisor, their least common multiple, and write their greatest common divisor in the form $ax + by$ for some integers $x$ and $y$.
	\begin{enumerate}
		\item[(a)] $a = 20$, $b = 13$.
		\item[(b)] $a = 69$, $b = 372$.
		\item[(c)] $a = 792$, $b = 275$.
		\item[(d)] $a = 11391$, $b = 5673$.
		\item[(e)] $a = 1761$, $b = 1567$.
		\item[(f)] $a = 507885$, $b = 60808$.
	\end{enumerate}
\end{exercise}

\begin{proof}[Solution]
	\begin{enumerate}
		\item[(a)] We first find the g.c.d.:
			\begin{align*}
				20 &= (1)(13) + 7 \\
				13 &= (1)(7) + 6 \\
				7 &= (1)(6) + 1 \\
				6 &= (6)(1)\,.
			\end{align*}
			So $\gcd(20,13) = 1$.
			
			We now find the l.c.m. We have $\gcd(a,b)\lcm(a,b) = ab$, and so
			\[
				\lcm(20,13) = \frac{20(13)}{\gcd(20,13)} = 260\,.
			\]
			
			We now write the g.c.d. in the form $ax + by$:
			\begin{align*}
				1 &= 7 - (1)(6) \\
					&= 7 - (1)[13 - (1)(7)] \\
					&= (2)7 - (13)1 \\
					&= (2)[20 - (1)13] - 13(1) \\
					&= (2)20 - (3)13
			\end{align*}
		\item[(b)] We first find the g.c.d.:
			\begin{align*}
				69 &= (0)(372) + 69 \\
				372 &= (5)(69) + 27 \\
				69 &= (2)(27) + 15 \\
				27 &= (1)(15) + 12 \\
				15 &= (1)(12) + 3 \\
				12 &= (4)(3)\,.
			\end{align*}
			So $\gcd(69,372) = 3$.
			
			We now find the l.c.m.:
			\[
				\lcm(69,372) = \frac{69(372)}{3} = 8556\,.
			\]
			
			We now write the g.c.d. in the form $ax + by$:
			\begin{align*}
				3 &= 15 - (1)(12) \\
					&= 15 - [27 - 15] \\
					&= (2)15 - 27 \\
					&= (2)[69 - 2(27)] - 27 \\
					&= (2)69 - 5(27) \\
					&= (2)69 - 5[372 - (5)69] \\
					&= (27)69 - (5)372\,.
			\end{align*}
		\item[(c)] We first find the g.c.d.:
			\begin{align*}
				792 &= 2(275) + 242 \\
				275 &= 1(242) + 33 \\
				242 &= 7(33) + 11 \\
				33 &= 3(11)\,.
			\end{align*}
			So $\gcd(792,275) = 11$.
			
			We now find the l.c.m.:
			\[
				\lcm(792,275) = \frac{792(275)}{11} = 19800\,.
			\]
			
			We now write the g.c.d. in the form $ax + by$:
			\begin{align*}
				11 &= 242 - 7(33) \\
					&= 242 - 7[275 - 242] \\
					&= (8)242 - (7)275 \\
					&= 8[792 - (2)275] - (7)275 \\
					&= (8)792 - (23)275\,.
			\end{align*}
		\item[(d)] We first find the g.c.d.:
			\begin{align*}
				11391 &= 2(5673) + 45 \\
				5673 &= 126(45) + 3 \\
				45 &= 15(3)\,.
			\end{align*}
			So $\gcd(11391,5673) = 3$.
			
			We now find the l.c.m.:
			\[
				\lcm(11391,5673) = \frac{11391(5673)}{3} = 21540381\,.
			\]
			
			We now write the g.c.d. in the form $ax + by$:
			\begin{align*}
				3 &= 5673 - 126(45) \\
					&= 5673 - 126[11391 - 2(5673)] \\
					&= (253)5673 - (126)11391\,.
			\end{align*}
		\item[(e)] We first find the g.c.d.:
			\begin{align*}
				1761 &= (1)1567 + 194 \\
				1567 &= (8)194 + 15 \\
				194 &= (12)15 + 14 \\
				15 &= (1)14 + 1 \\
				14 &= (14)1\,.
			\end{align*}
			So $\gcd(1761,1567) = 1$.
			
			We now find the l.c.m.:
			\[
				\lcm(1761,1567) = \frac{1761(1567)}{1} = 2759487\,.
			\]
			
			We now write the g.c.d. in the form $ax + by$:
			\begin{align*}
				1 &= 15 - 14 \\
					&= 15 - [194 - (12)15] \\
					&= (13)15 - 194 \\
					&= (13)[1567 - (8)194] - 194 \\
					&= (13)1567 - (105)194 \\
					&= (13)1567 - (105)[1761 - 1567] \\
					&= (118)1567 - (105)1761\,.
			\end{align*}
		\item[(f)] We first find the g.c.d.:
			\begin{align*}
				507885 &= (8)60808 + 21421 \\
				60808 &= (2)21421 + 17966 \\
				21421 &= (1)17966 + 3455 \\
				17966 &= (5)3455 + 691 \\
				3455 &= (5)691\,.
			\end{align*}
			So $\gcd(507885,60808) = 691$.
			
			We now find the l.c.m.:
			\[
				\lcm(507885,60808) = \frac{507885(60808)}{691} = 44693880\,.
			\]
			
			We now write the g.c.d. in the form $ax + by$:
			\begin{align*}
				691 &= 17966 - (5)3455 \\
					&= 17966 - (5)[21421 - 17966] \\
					&= (6)17966 - (5)21421 \\
					&= (6)[60808 - (2)21421] - (5)21421 \\
					&= (6)60808 - (17)21421 \\
					&= (6)60808 - (17)[507885 - (8)60808] \\
					&= (142)60808 - (17)507885\,.
			\end{align*}
	\end{enumerate}
\end{proof}

%%%%%%%%%%%%%%%%%%%%%%%%%%%%%%%%%%%%%%%%%%%%%%%%%%%%%%%%%%%%%%%%%%

\begin{exercise}
	Prove that if the integer $k$ divides the integers $a$ and $b$ then $k$ divides $as + bt$ for every pair of integers $s$ and $t$.
\end{exercise}

\begin{proof}
	Suppose $k \mid a$ and $k \mid b$. Then $a = km$ and $b = nk$ for some integers $m$ and $n$. So $as + bt = mks + nkt = k(ms + nt)$, and therefore $k \mid (as + bt)$.
\end{proof}

%%%%%%%%%%%%%%%%%%%%%%%%%%%%%%%%%%%%%%%%%%%%%%%%%%%%%%%%%%%%%%%%%%

\begin{exercise}
	Prove that if $n$ is composite then there are integers $a$ and $b$ such that $n$ divides $ab$ but $n$ does not divide either $a$ or $b$.
\end{exercise}

\begin{proof}
	Suppose $n$ is composite. Then $n = ab$ for some $a$ and $b$ with $|a| < n$ and $|b| < n$. Since $n = ab$, $n \mid ab$, but $n \nmid a$ and $n \nmid b$ since $|a| < n$ and $|b| < n$.
\end{proof}

%%%%%%%%%%%%%%%%%%%%%%%%%%%%%%%%%%%%%%%%%%%%%%%%%%%%%%%%%%%%%%%%%%

\begin{exercise}
	Let $a,b$ and $N$ be fixed integers with $a$ and $b$ nonzero and let $d = \gcd(a,b)$ be the greatest common divisor of $a$ and $b$. Suppose $x_0$ and $y_0$ are particular solutions to $ax + by = N$ (i.e., $ax_0 + by_0 = N$). Prove for any integer $t$ that the integers
	\[
		x = x_0 + \frac{b}{d}t
		\quad
		\text{and}
		\quad
		y = y_0 - \frac{a}{d}t
	\]
	are also solutions to $ax + by = N$ (this is in fact the general solution).
\end{exercise}

\begin{proof}
	Suppose $ax + by = N$ has the particular solutions $x_0$ and $y_0$; that is, suppose $ax_0 + by_0 = N$. Let $t$ be arbitrary. Then
	\[
		a \left( x_0 + \frac{b}{d}t \right) +
		b \left( y_0 - \frac{a}{d}t \right) 
			= ax_0 + \frac{ab}{d}t + by_0 - \frac{ab}{d}t
			= N\,.
	\]
\end{proof}

%%%%%%%%%%%%%%%%%%%%%%%%%%%%%%%%%%%%%%%%%%%%%%%%%%%%%%%%%%%%%%%%%%

\begin{exercise}
	Determine the value $\varphi(n)$ for each integer $n \leq 30$ where $\varphi$ denotes the Euler $\varphi$-function.
\end{exercise}

\begin{proof}[Solution]
	We have $\varphi(1) = 1$, $\varphi(2) = 1$, $\varphi(3) = 2$, $\varphi(4) = 2$, $\varphi(5) = 4$, $\varphi(6) = 2$, $\varphi(7) = 6$, $\varphi(8) = \varphi(2^3) = 2^{3-1}(2-1) = 4$, $\varphi(9) = \varphi(3^2) = 3(3-1) = 6$, $\varphi(10) = \varphi(2)\varphi(5) = 4$, $\varphi(11) = 10$, $\varphi(12) = \varphi(3)\varphi(4) = 2(2) = 4$, $\varphi(13) = 12$, $\varphi(14) = \varphi(2)\varphi(7) = 6$, $\varphi(15) = \varphi(3)\varphi(5) = 2(4) = 8$, $\varphi(16) = \varphi(2^4) = 2^3(2-1) = 8$, $\varphi(17) = 16$, $\varphi(18) = \varphi(2)\varphi(9) = 6$, $\varphi(19) = 18$, $\varphi(20) = \varphi(4)\varphi(5) = 8$, $\varphi(21) = \varphi(3)\varphi(7) = 12$, $\varphi(22) = \varphi(2)\varphi(11) = 10$, $\varphi(23) = 22$, $\varphi(24) = \varphi(3)\varphi(8) = 8$, $\varphi(25) = \varphi(5^2) = 5(4) = 20$, $\varphi(26) = \varphi(2)\varphi(13) = 12$, $\varphi(27) = \varphi(3^3) = 3^2(2) = 18$, $\varphi(28) = \varphi(4)\varphi(7) = 12$, $\varphi(29) = 28$, $\varphi(30) = \varphi(3)\varphi(10) = 8$.
\end{proof}

%%%%%%%%%%%%%%%%%%%%%%%%%%%%%%%%%%%%%%%%%%%%%%%%%%%%%%%%%%%%%%%%%%

\begin{exercise}
	Prove the Well Ordering Property of $\integers$ by induction and prove the minimal element is unique.
\end{exercise}

\begin{proof}
	Let $A$ be an arbitrary nonempty subset of $\zplus$.
	
	Base Case: $A$ has a single element. Then this element is the minimal element of $A$, for $m \leq m$. Moreover, being the only element, it is unique.
	
	Inductive Step: Now suppose $A$ has $n$ elements, and the theorem holds for any set of $n$ or fewer elements. We show the theorem holds for a set of $n + 1$ elements. If new element is smaller than existing minimal element, it is the minimal element. Otherwise, the existing minimal element remains the minimal element.
\end{proof}

%%%%%%%%%%%%%%%%%%%%%%%%%%%%%%%%%%%%%%%%%%%%%%%%%%%%%%%%%%%%%%%%%%

\begin{exercise}
	If $p$ is a prime prove that there do not exist nonzero integers $a$ and $b$ such that $a^2 = pb^2$ (i.e., $\sqrt{p}$ is not a rational number).
\end{exercise}

\begin{proof}
	By way of contradiction, suppose $p$ is prime and there exists 
	$a,b$ such that $a^2 = pb^2$; that is, $\sqrt{p} = 
	\frac{a}{b}$. Assume $\gcd(a,b) = 1$; that is, suppose 
	$\sqrt{p}$ is written in its most reduced form. Then $p \mid 
	a^2$, and so $p \mid a$. That is, $a = pc$ for some integer 
	$c$. Then $a^2 = p^2c^2 = pb^2$. And so, $pc^2 = b^2$, and we 
	see that $p \mid b^2$ and so $p \mid b$. Since $p \mid a$ and 
	$p \mid b$, then $p \mid ab$, contradicting the fact that 
	$\gcd(a,b) = 1$.
\end{proof}

%%%%%%%%%%%%%%%%%%%%%%%%%%%%%%%%%%%%%%%%%%%%%%%%%%%%%%%%%%%%%%%%%%

\begin{exercise}
	Let $p$ be a prime, $n \in \zplus$. Find a formula for the largest power of $p$ which divides $n! = n(n-1)(n-2) \cdots 2 \cdot 1$ (it involves the greatest integer function).
\end{exercise}

\begin{proof}[Solution]
	There are $\left\lfloor \frac{n}{p^k} \right\rfloor$ multiples 
	of $p^k$ in the list $1, 2, \dots, n$; that is, $p^k$ divides 
	$\left\lfloor \frac{n}{p^k} \right\rfloor$ numbers less than or 
	equal to $n$. Thus, $\sum_k \left\lfloor \frac{n}{p^k} 
	\right\rfloor$ accounts for the highest order of $p$ which 
	divides $n!$.
\end{proof}

%%%%%%%%%%%%%%%%%%%%%%%%%%%%%%%%%%%%%%%%%%%%%%%%%%%%%%%%%%%%%%%%%%

\begin{exercise}
	Write a computer program to determine the greatest common divisor $\gcd(a,b)$ of two integers $a$ and $b$ and to express $\gcd(a,b)$ in the form $ax + by$ for some integers $x$ and $y$.
\end{exercise}

\begin{proof}[Solution]
	\lstinputlisting[language=python]{Solutions/gcd.py}
\end{proof}

%%%%%%%%%%%%%%%%%%%%%%%%%%%%%%%%%%%%%%%%%%%%%%%%%%%%%%%%%%%%%%%%%%

\begin{exercise}
	Prove for any given positive integer $N$ there exist only finitely many integers $n$ with $\varphi(n) = N$ where $\varphi$ denotes Euler's $\varphi$-function. Conclude in particular that $\varphi(n)$ tends to infinity as $n$ tends to infinity.
\end{exercise}

\begin{proof}
	Let $N \in \zplus$ be arbitrary. For any $n = p_1^{\alpha_1} 
	\cdots p_s^{\alpha_s}$ such that $\varphi(n)  = p_1^{\alpha_1 - 
	1}(p_1 - 1) \cdots p_s^{\alpha_s - 1}(p_s - 1) = N$, it is 
	clear from this formula that for any prime factor $p_i$ of $n$, 
	we have $\varphi(p_i) = p_i - 1 \mid N$. Thus $p_i - 1 \leq N$, 
	or $p_i \leq N + 1$.
	
	Therefore, let $p_1, \cdots, p_t$ be the primes which are less 
	than or equal to $N + 1$. All $n$ such that $\varphi(n) = N$ 
	thus have prime factorizations of the form $p_1^{\alpha_1} 
	\cdots p_t^{\alpha_t}$. Moreover, for $1 \leq i \leq t$, we 
	have $p_i^{\alpha_i - 1} \mid N$. Let $k_i$ be the largest 
	integer such that $p_i^{k_i} \mid N$. Obviously, $\alpha_i \leq 
	k_i + 1$, and so there are at most $\prod_i (k_i + 1)$ integers 
	$n$ such that $\varphi(n) = N$.
\end{proof}

%%%%%%%%%%%%%%%%%%%%%%%%%%%%%%%%%%%%%%%%%%%%%%%%%%%%%%%%%%%%%%%%%%

\begin{exercise}
	Prove that if $d$ divides $n$ then $\varphi(d)$ divides $\varphi(n)$ where $\varphi$ denotes Euler's $\varphi$-function.
\end{exercise}

\begin{proof}
	Suppose $d \mid n$, and let the prime factorizations of $n$ and 
	$d$ be $n = p_1^{\alpha_1} p_2^{\alpha_2} \cdots 
	p_s^{\alpha_s}$ and $d = p_1^{\beta_1} p_2^{\beta_2} \cdots 
	p_s^{\beta_s}p_{s+1}^{\beta_{s+1}} \cdots p_t^{\beta_t}$, where 
	$\beta_i \geq \alpha_i$ for $1 \leq i \leq s$.
	
	We have
	\[
		\varphi(n) = p_1^{\alpha_1 - 1}(p_1 - 1) \cdots 
		p_s^{\alpha_s - 1}(p_s - 1)\,,
	\]
	and therefore
	\begin{align*}
		\varphi(d) &= p_1^{\beta_1 - 1}(p_1 - 1) \cdots p_s^{\beta_s
			- 1}(p_s - 1) p_{s+1}^{\beta_{s+1} - 1}(p_{s+1} - 1) 
			\cdots p_t^{\beta_t - 1}(p_t - 1) \\
		&= p_1^{\alpha_1 - 1} p_1^{\gamma_1} (p_1 - 1) \cdots 
			p_s^{\alpha_s - 1} p_1^{\gamma_s} (p_s - 1)
			p_{s+1}^{\beta_{s+1} - 1}(p_{s+1} - 1) \cdots 
			p_t^{\beta_t - 1}(p_t - 1) \\
		&= \varphi(n) p_1^{\gamma_1} \cdots p_s^{\gamma_s} 
			p_{s+1}^{\beta_{s+1} - 1}(p_{s+1} - 1) \cdots 
			p_t^{\beta_t - 1}(p_t - 1)\,.
	\end{align*}	
	And so we have that $\varphi(n) \mid \varphi(d)$.
\end{proof}

%%%%%%%%%%%%%%%%%%%%%%%%%%%%%%%%%%%%%%%%%%%%%%%%%%%%%%%%%%%%%%%%%%
%%%%%%%%%%%%%%%%%%%%%%%%%%%%%%%%%%%%%%%%%%%%%%%%%%%%%%%%%%%%%%%%%%

\subsection{$\integers/n\integers$: The Integers Modulo $n$}

\begin{exercise}
	Write down explicitly all the elements in the residue classes of $\integers/18\integers$.
\end{exercise}

%%%%%%%%%%%%%%%%%%%%%%%%%%%%%%%%%%%%%%%%%%%%%%%%%%%%%%%%%%%%%%%%%%

\begin{exercise}
	Prove that the distinct equivalence classes in $\integers/n\integers$ are precisely $\bar{0}, \bar{1}, \bar{2}, \dots, \overline{n-1}$ (use the Division Algorithm).
\end{exercise}

%%%%%%%%%%%%%%%%%%%%%%%%%%%%%%%%%%%%%%%%%%%%%%%%%%%%%%%%%%%%%%%%%%

\begin{exercise}
	Prove that if $a = a_n 10^n + a_{n-1} 10^{n-1} + \cdots + a_1 10 + a_0$ is any positive integer then $a \equiv_9 a_n + a_{n-1} + \cdots a_1 + a_0$ (note that this is the usual arithmetic rule that the remainder after division by 9 is the same as the sum of the decimal digits mod 9--in particular an integer is divisible by 9 if and only if the sum of its digits is divisible by 9) [note that $10 \equiv_9 1$].
\end{exercise}

%%%%%%%%%%%%%%%%%%%%%%%%%%%%%%%%%%%%%%%%%%%%%%%%%%%%%%%%%%%%%%%%%%

\begin{exercise}
	Compute the remainder when $37^{100}$ is divided by 29.
\end{exercise}

%%%%%%%%%%%%%%%%%%%%%%%%%%%%%%%%%%%%%%%%%%%%%%%%%%%%%%%%%%%%%%%%%%

\begin{exercise}
	Compute the last two digits of $9^{1500}$.
\end{exercise}

%%%%%%%%%%%%%%%%%%%%%%%%%%%%%%%%%%%%%%%%%%%%%%%%%%%%%%%%%%%%%%%%%%

\begin{exercise}
	Prove that the squares of the elements in $\integers/4\integers$ are just $\bar{0}$ and $\bar{1}$.
\end{exercise}

%%%%%%%%%%%%%%%%%%%%%%%%%%%%%%%%%%%%%%%%%%%%%%%%%%%%%%%%%%%%%%%%%%

\begin{exercise}
	Prove for any integers $a$ and $b$ that $a^2 + b^2$ never leaves a remainder of 3 when divided by 4 (use the previous exercise).
\end{exercise}

%%%%%%%%%%%%%%%%%%%%%%%%%%%%%%%%%%%%%%%%%%%%%%%%%%%%%%%%%%%%%%%%%%

\begin{exercise}
	Prove that the equation $a^2 + b^2 = 3c^2$ has no solutions in nonzero integers $a$, $b$, and $c$. [Consider the equation mod 4 as in the previous two exercises and show that $a$, $b$, and $c$ would all have to be divisible by 2. Then each of $a^2$, $b^2$, and $c^2$ has a factor of 4 and by dividing through by 4 show that there would be a smaller set of solutions to the original equation. Iterate to reach a contradiction.]
\end{exercise}

%%%%%%%%%%%%%%%%%%%%%%%%%%%%%%%%%%%%%%%%%%%%%%%%%%%%%%%%%%%%%%%%%%

\begin{exercise}
	Prove that the square of any odd integer always leaves a remainder of 1 when divided by 8.
\end{exercise}

%%%%%%%%%%%%%%%%%%%%%%%%%%%%%%%%%%%%%%%%%%%%%%%%%%%%%%%%%%%%%%%%%%

\begin{exercise}
	Prove that the number of elements of $(\integers/n\integers)^\times$ is $\varphi(n)$ where $\varphi$ denotes the Euler $\varphi$-function.
\end{exercise}

%%%%%%%%%%%%%%%%%%%%%%%%%%%%%%%%%%%%%%%%%%%%%%%%%%%%%%%%%%%%%%%%%%

\begin{exercise}
	Prove that if $\bar{a}, \bar{b} \in (\integers/n\integers)^\times$, then $\bar{a} \cdot \bar{b} \in (\integers/n\integers)^\times$.
\end{exercise}

%%%%%%%%%%%%%%%%%%%%%%%%%%%%%%%%%%%%%%%%%%%%%%%%%%%%%%%%%%%%%%%%%%

\begin{exercise}
	Let $n \in \integers$, $n > 1$, and let $a \in \integers$ with $1 \leq a \leq n$. Prove if $a$ and $n$ are not relatively prime, there exists an integer $b$ with $1 \leq b < n$ such that $ab \equiv_n 0$ and deduce that there cannot be an integer $c$ such that $ac \equiv_n 1$.
\end{exercise}

%%%%%%%%%%%%%%%%%%%%%%%%%%%%%%%%%%%%%%%%%%%%%%%%%%%%%%%%%%%%%%%%%%

\begin{exercise}
	Let $n \in \integers$, $n > 1$, and let $a \in \integers$ with $1 \leq a \leq n$. Prove that if $a$ and $n$ are relatively prime then there is an integer $c$ such that $ac \equiv_n 1$ [use the fact that the g.c.d. of two integers is a $\integers$-linear combination of the integers].
\end{exercise}

%%%%%%%%%%%%%%%%%%%%%%%%%%%%%%%%%%%%%%%%%%%%%%%%%%%%%%%%%%%%%%%%%%

\begin{exercise}
	Conclude from the previous two exercises that $(\integers/n\integers)^\times$ is the set of elements $\bar{a}$ of $\integers/n\integers$ with $\gcd(a,n) = 1$ and hence prove Proposition 4. Verify this directly in the case $n = 12$.
\end{exercise}

%%%%%%%%%%%%%%%%%%%%%%%%%%%%%%%%%%%%%%%%%%%%%%%%%%%%%%%%%%%%%%%%%%

\begin{exercise}
	For each of the following pairs of integers $a$ and $n$, show that $a$ is relatively prime to $n$ and determine the multiplicative inverse of $\bar{a}$ in $\integers/n\integers$.
	\begin{enumerate}
		\item[(a)] $a = 13$, $n = 20$.
		\item[(b)] $a = 69$, $n = 89$.
		\item[(c)] $a = 1891$, $n = 3797$.
		\item[(d)] $a = 6003722857$, $n = 77695236973$. [The Euclidean Algorithm requires on 3 steps for these integers.]
	\end{enumerate}
\end{exercise}

%%%%%%%%%%%%%%%%%%%%%%%%%%%%%%%%%%%%%%%%%%%%%%%%%%%%%%%%%%%%%%%%%%

\begin{exercise}
	Write a computer program to add and multiply mod $n$, for any $n$ given as input. The output of these operations should be the least residues of the sums and products of two integers. Also include the feature that if $\gcd(a,n) = 1$, an integer between 1 and $n - 1$ such that $\bar{a} \cdot \bar{c} = \bar{1}$ may be printed on request. (Your program should not, of course, simply quote ``mod" functions already built into many systems.)
\end{exercise}

%%%%%%%%%%%%%%%%%%%%%%%%%%%%%%%%%%%%%%%%%%%%%%%%%%%%%%%%%%%%%%%%%%
%%%%%%%%%%%%%%%%%%%%%%%%%%%%%%%%%%%%%%%%%%%%%%%%%%%%%%%%%%%%%%%%%%

\section{Introduction to Groups}

\subsection{Basic Axioms and Examples}

Let $G$ be a group.

\begin{exercise}
	Determine which of the following binary operations are 
	associative: 
	\begin{enumerate}
		\item[(a)] the operation $\cdot$ on $\integers$ defined by 
		$a \cdot b = a - b$
		
		\item[(b)] the operation $\cdot$ on $\reals$ defined by $a 
		\cdot b = a + b + ab$
		
		\item[(c)] the operation $\cdot$ on $\rationals$ defined by 
		$a \cdot b = \dfrac{a+b}{5}$
		
		\item[(d)] the operation $\cdot$ on $\integers \times 
		\integers$ defined by $(a,b) \cdot (c,d) = (ad + bc, bd)$
		
		\item[(e)] the operation $\cdot$ on $\rationals \setminus 
		\{ 0 \}$ defined by $a \cdot b = \dfrac{a}{b}$\,.
	\end{enumerate}
\end{exercise}

%%%%%%%%%%%%%%%%%%%%%%%%%%%%%%%%%%%%%%%%%%%%%%%%%%%%%%%%%%%%%%%%%%

\begin{exercise}
	Decide which of the binary operations in the preceding exercise 
	are commutative.
\end{exercise}

%%%%%%%%%%%%%%%%%%%%%%%%%%%%%%%%%%%%%%%%%%%%%%%%%%%%%%%%%%%%%%%%%%

\begin{exercise}
	Prove that addition of residue classes in 
	$\integers/n\integers$ is associative (you may assume it is 
	well defined).
\end{exercise}

%%%%%%%%%%%%%%%%%%%%%%%%%%%%%%%%%%%%%%%%%%%%%%%%%%%%%%%%%%%%%%%%%%

\begin{exercise}
	Prove that multiplication of residue classes in 
	$\integers/n\integers$ is associative (you may assume it is 
	well defined).
\end{exercise}

%%%%%%%%%%%%%%%%%%%%%%%%%%%%%%%%%%%%%%%%%%%%%%%%%%%%%%%%%%%%%%%%%%

\begin{exercise}
	Prove for all $n > 1$ that $\integers/n\integers$ is not a 
	group under multiplication of residue classes.
\end{exercise}

%%%%%%%%%%%%%%%%%%%%%%%%%%%%%%%%%%%%%%%%%%%%%%%%%%%%%%%%%%%%%%%%%%

\begin{exercise}
	Determine which of the following sets are groups under addition:
	\begin{enumerate}
		\item[(a)] the set of rational numbers (including $0=0/1$) 
		in lowest terms whose denominators are odd
		
		\item[(b)] the set of rational numbers (including $0=0/1$) 
		in lowest terms whose denominators are even
		
		\item[(c)] the set of rational numbers of absolute value < 1
		
		\item[(d)] the set of rational numbers of absolute value 
		$\geq 1$ together with 0
		
		\item[(e)] the set of rational numbers with denominators 
		equal to 1 or 2
		
		\item[(f)] the set of rational numbers with denominators 
		equal to 1, 2, or 3.
	\end{enumerate}
\end{exercise}

%%%%%%%%%%%%%%%%%%%%%%%%%%%%%%%%%%%%%%%%%%%%%%%%%%%%%%%%%%%%%%%%%%

\begin{exercise}
	Let $G = \{ x \in \reals \mid 0 \leq x < 1 \}$ and for $x,y \in 
	G$ let $x \cdot y$ be the fractional part of $x+y$ (i.e., $x 
	\cdot y = x + y - \lfloor x + y \rfloor$ where $\lfloor a 
	\rfloor$ is the greatest integer less than or equal to $a$). 
	Prove that $\cdot$ is a well 
	defined binary operation on $G$ and that $G$ is an abelian 
	group under $\cdot$ (called the \df{real numbers mod 1}).
\end{exercise}

%%%%%%%%%%%%%%%%%%%%%%%%%%%%%%%%%%%%%%%%%%%%%%%%%%%%%%%%%%%%%%%%%%

\begin{exercise}
	Let $G = \{ z \in \complex \mid z^n = 1 \text{ for some } n \in 
	\zplus \}$.
	\begin{enumerate}
		\item[(a)] Prove that $G$ is a group under multiplication 
		(called the group of \df{roots of unity} in $\complex$).
		
		\item[(b)] Prove that $G$ is not a group under addition.
	\end{enumerate}
\end{exercise}

%%%%%%%%%%%%%%%%%%%%%%%%%%%%%%%%%%%%%%%%%%%%%%%%%%%%%%%%%%%%%%%%%%

\begin{exercise}
	Let $G = \{ a + b\sqrt{2} \in \reals \mid a,b \in \rationals 
	\}$.
	\begin{enumerate}
		\item[(a)] Prove that $G$ is a group under addition.
		
		\item[(b)] Prove that the nonzero elements of $G$ are a 
		group under multiplication. [``Rationalize the 
		denominators'' to find multiplicative inverses.]
	\end{enumerate}
\end{exercise}

%%%%%%%%%%%%%%%%%%%%%%%%%%%%%%%%%%%%%%%%%%%%%%%%%%%%%%%%%%%%%%%%%%

\begin{exercise}
	Prove that a finite group is abelian if and only if its group 
	table is a symmetric matrix.
\end{exercise}

%%%%%%%%%%%%%%%%%%%%%%%%%%%%%%%%%%%%%%%%%%%%%%%%%%%%%%%%%%%%%%%%%%

\begin{exercise}
	Find the orders of each element of the additive group 
	$\integers/12\integers$.
\end{exercise}

%%%%%%%%%%%%%%%%%%%%%%%%%%%%%%%%%%%%%%%%%%%%%%%%%%%%%%%%%%%%%%%%%%

\begin{exercise}
	Find the orders of the following elements of the multiplicative 
	group $(\integers/12\integers)^\times$: $\overline{1}$, 
	$\overline{-1}$, $\overline{5}$, $\overline{7}$, 
	$\overline{-7}$, $\overline{13}$
\end{exercise}































